\documentclass[runningheads,a4paper]{llncs}

\usepackage{amssymb}
\setcounter{tocdepth}{3}
\usepackage{graphicx}

\usepackage{url}
\urldef{\mailsa}\path|{felix.kurth, schupp}@tu-harburg.de|
\newcommand{\keywords}[1]{\par\addvspace\baselineskip
\noindent\keywordname\enspace\ignorespaces#1}
\newcommand{\UMLType}[1]{\textsf{\textit{#1}}}
\newcommand{\TCGType}[1]{#1}
\newcommand{\UMLReference}[1]{\textsf{\textit{#1}}} 

\begin{document}

\mainmatter  % start of an individual contribution

% first the title is needed
\title{Automated Generation of Unit Tests from UML Activity Diagrams using the AMPL Interface for Constraint Solvers}

% a short form should be given in case it is too long for the running head
\titlerunning{Automated Generation of Unit Tests}

% the name(s) of the author(s) follow(s) next
%
% NB: Chinese authors should write their first names(s) in front of
% their surnames. This ensures that the names appear correctly in
% the running heads and the author index.
%
\author{Felix Kurth%
% \thanks{Please note that the LNCS Editorial assumes that all authors have used
% the western naming convention, with given names preceding surnames. This determines
% the structure of the names in the running heads and the author index.}%
\and Sibylle Schupp}
%
%\authorrunning{Lecture Notes in Computer Science: Authors' Instructions}
% (feature abused for this document to repeat the title also on left hand pages)

% the affiliations are given next; don't give your e-mail address
% unless you accept that it will be published
\institute{Hamburg University of Technology, Institute for Software Systems,\\
Schwarzenbergstr. 95, 21073 Hamburg, Germany\\
\mailsa\\
\url{http://sts.tu-harburg.de}}

%
% NB: a more complex sample for affiliations and the mapping to the
% corresponding authors can be found in the file "llncs.dem"
% (search for the string "\mainmatter" where a contribution starts).
% "llncs.dem" accompanies the document class "llncs.cls".
%

%\toctitle{Lecture Notes in Computer Science}
%\tocauthor{Authors' Instructions}
\maketitle%

\begin{abstract}
The abstract should summarise the contents of the paper and should
contain at least 70 and at most 150 words. It should be written using the
\emph{abstract} environment.
\keywords{Model-Based Testing, Constraint Solving, Symbolic Execution, Boundary Value Analysis}
\end{abstract}


\section{Introduction}
From the domain of Model-Based Engineering arises the wish to generate unit tests for functions/operations from their design model. We present a transformation from an activity diagram into an `A Mathematical Programming Language' (AMPL) model. The solution of an AMPL program contains input values causing a certain control flow path to be executed. The oracle value is also computed.\\
The presented transformation is implemented in a proof-of-concept fully automated test generation tool. It is available as Eclipse Plug-in at \cite{PartegWebsite}\\
We used depth first search and to find control flow paths for which we generate test data. We also implemented early infeasible path elimination to reduce the runtime of our algorithm.
\subsection{Literature Review}
\subsection{Overview}
\section{Preliminaries}
\subsection{Semantics of the Test Model}
As test model we assume an activity diagram with \UMLType{Action}s and \UMLType{ControlFlow}s modelling the control flow of an \UMLType{Operation}. The \UMLType{Activity} is linked as \UMLReference{method} to its specifying \UMLType{Operation}.
\subsection{AMPL and its Solvers}
\section{The Algorithm}
\subsection{AMPL Transformation}
\subsection{Boundary Value Analysis}
\subsection{Early Infeasible Path Elimination}
\section{Results}
\subsection{Proof--of--Concept Tool}
\subsection{Case Study}
\section{Outlook}

\bibliographystyle{splncs}
\bibliography{../Thesis/bibtex}
\end{document}